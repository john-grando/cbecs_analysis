\section*{Research}
\label{sec:literature_review}
\addcontentsline{toc}{section}{\nameref{sec:literature_review}}

\subsection{Related Work}
The idea of determing building energy efficiency is not a novel concept in itself.  As previously mentioned, ENERGY STAR has a building benchmarking tool\footnote{\href{https://www.energystar.gov/buildings/about-us/how-can-we-help-you/benchmark-energy-use/benchmarking }{\url{https://www.energystar.gov/buildings/about-us/how-can-we-help-you/benchmark-energy-use/benchmarking}}}.  Additionally, the United States Green Building Council has created the Arc Platform\footnote{\href{https://arcskoru.com/}{https://arcskoru.com/}} which provides benchmarking and active monitoring features.  While these platforms provide building comparisons in the form of an overall score, it is difficult to explore the space around the building attribute inputs themselves as well as compare consumption to randomly sampled buidlings.  With this functionality, a more direct comparison can be made and relative environmental impact can be measured.

\subsection{Literature Review}

