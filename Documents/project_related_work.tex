\section*{Research}
\label{sec:literature_review}
\addcontentsline{toc}{section}{\nameref{sec:literature_review}}

\subsection{Related Work}
The idea of determing building energy efficiency is not a novel concept in itself.  As previously mentioned, ENERGY STAR has a building benchmarking tool\footnote{\href{https://www.energystar.gov/buildings/about-us/how-can-we-help-you/benchmark-energy-use/benchmarking }{\url{https://www.energystar.gov/buildings/about-us/how-can-we-help-you/benchmark-energy-use/benchmarking}}}.  Additionally, the United States Green Building Council has created the Arc Platform\footnote{\href{https://arcskoru.com/}{\url{https://arcskoru.com/}}} which provides benchmarking and active monitoring features.  While these platforms provide building comparisons in the form of an overall score, it is difficult to explore the space around the building attribute inputs themselves as well as compare consumption of a specific building to its equivalent standard practice building.  With this functionality, a more direct comparison can be made and relative environmental impact can be measured.

\subsection{Literature Review}

There are a variety of texts that are dedicated to the analysis of building energy consumption, and determining operating efficiency, such as ASHRAE Guildine 14.  Also, there are guidelines that must be followed for buildings undergoing new construction or major renovation, which have energy compliance sections (ASHRAE Guideline 90.1, 189.1, International Energy Conservation Code).  Particularly, there is a well thought out process for auditing commercial buildings, known as ASHRAE Audits, which start at the lowest level (I) and progress to the highest level (III) as the opportunity for energy and cost savings becomes more apparent \footnote{\href{http://aea.us.org/3143-2.html}{\url{http://aea.us.org/3143-2.html}}}.  