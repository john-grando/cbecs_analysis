\section*{Introduction}
\label{sec:introduction}
\addcontentsline{toc}{section}{\nameref{sec:introduction}}

The concept of evaluating building performance typically requires two things; measuring the property in question's consumption, and comparing it to a standard practice equivalent.  A baseline building consumption value is useful to inform people of expected opereational performance; however, it is only useful if it is accurate.  Additionally, there are benefits to using simple, available, predictors, which can make it easier to actually use the model.

Building owners, local governments, and utility providers are all looking for ways to reduce energy consumption.  Reasons for doing so can vary all the way from social responsibility to economic gain.  Some people want to show off an efficient building, others want to identify propertie that are in need of improvement.  however, in order to do this, a standard practice baseline value must be determined.  Additionally, in order for the final product to be useful, it is important that the final set of predictors be parsimonious and be realistically available to users.

Comparing summary statistics between buildings, such as energy use per square foot, is not as simple as it seems because there are a multitude of factors that affect a building’s energy consumption profile.  The building use type can cause energy use to vary by a large amount; such as office buildings and refrigerated warehouses.  Also, seemingly small factors, such as the hours of operation, may have significant impacts as well.  This complexity of making similar comparisons creates a situation where it is difficult to determine whether a building is performing consistent with, or better than, other standard practice buildings.

Commercially, using the most popular example, ENERGY STAR \footnote{\href{https://www.energystar.gov/}{ENERGY STAR - \url{https://www.energystar.gov/}}} has implemented a benchmarking algorithm that scores buildings on a scale from 1 – 100 using market-available data.  The output of this benchmarking algorithm is a unit-less score, as well as a reference ‘baseline’ building; however, the methodology is not released and it is unclear what factors are important to influence the energy consumption of the building.  These barriers make it difficult to provide custom comparisons and nearly impossible to make batch predictions from a set of buildings, or variations of buildings.

Every few years, the U.S. Energy Information Administration (EIA) conducts a survey attempting to record pertinent features of these buildings, known officially as the Commercial Buildings Energy Consumption Survey (CBECS)\footnote{\href{https://www.eia.gov/consumption/commercial/data/2012/index.php?view=microdata}{EIA Microdata}}.  While the survey is expansive (i.e. more than 600 tracked features), it is essential to create a model that is usable and only requires predictors that can easily be attained by building operators.  Therefore, in this study a series of models will be evaluated in order to determine the most important predictors which will then be used to train a final, more complex, model.  After completion of the model, the predictors will be evaluated on how easy they are to attain, and will possibly be exchanged with simpler variables that are highly correlated.