{\section*{Introduction}}
\noindent
what is the problem\\
The concept of evaluating building performance typically requires two things; measuring the property in question's consumption, and comparing it to a standard practice equivalent.  A baseline building consumption value is useful to inform people of expected opereational performance; however, it is only useful if it is accurate.  Additionally, there are benefits to using simple, available, predictors, which can make it easier to actually use the model.\\

why is it interesting and important\\
Building owners, local governments, and utility providers are all looking for ways to reduce energy consumption.  Reasons for doing so can vary all the way from social responsibility to economic gain.  Some people want to show off an efficient building, others want to identify propertie that are in need of improvement.  however, in order to do this, a standard practice baseline value must be determined.  Additionally, in order for the final product to be useful, it is important that the final set of predictors be parsimonious and be realistically available to users.\\

why is it hard (why naive approaches failed)\\
Comparing summary statistics between buildings, such as energy use per square foot, is not as simple as it seems because there are a multitude of factors that affect a building’s energy consumption profile.  The building use type can cause energy use to vary by a large amount; such as office buildings and refrigerated warehouses.  Also, seemingly small factors, such as the hours of operation, may have significant impacts as well.  This complexity of making similar comparisons creates a situation where it is difficult to determine whether a building is performing consistent with, or better than, other standard practice buildings.\\

why hasnt it been solved before or whats wrong with the previous solution, why does it differ?\\


what are the key components of my approach and results, include limitations\\
Every few years, the U.S. Energy Information Administration (EIA) conducts a survey attempting to record pertinent features of these buildings, known officially as the Commercial Buildings Energy Consumption Survey (CBECS)\footnote{\href{https://www.eia.gov/consumption/commercial/data/2012/index.php?view=microdata}{microdata - \url{https://www.eia.gov/consumption/commercial/data/2012/index.php?view=microdata}}}.  While the survey is expansive (i.e. more than 600 tracked features), it is best to create a model using the least amount of predictors possible, without losing significant predictive power, in order to make the final product usable when trying to apply it in practice.\\