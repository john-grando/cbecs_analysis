{\section*{Introduction}}
\noindent
what is the problem\\
Comparing summary statistics between buildings, such as energy use per square foot, is not as simple as it seems because there are a multitude of factors that affect a building’s energy consumption profile.  The complexity of making similar comparisons creates a situation where it is difficult to determine whether a building is performing consistent with, or better than, other standard practice buildings.  However, every few years, the U.S. Energy Information Administration (EIA) conducts a survey attempting to record pertinent features of these buildings, known officially as the Commercial Buildings Energy Consumption Survey (CBECS)\footnote{\href{https://www.eia.gov/consumption/commercial/data/2012/index.php?view=microdata}{microdata - \url{https://www.eia.gov/consumption/commercial/data/2012/index.php?view=microdata}}}.  While the survey is expansive (i.e. more than 600 tracked features), it is best to create a model using the least amount of predictors possible, without losing significant predictive power, in order to make the final product usable when trying to apply it in practice.\\
why is it interesting and important\\
why is it hard (why naive approaches failed)\\
why hasnt it been solved before or whats wrong with the previous solution, why does it differ?\\
what are the key components of my approach and results, include limitations\\