\section{Theory and Hypothesis}
\label{sec:theory_and_hypothesis}
%\addcontentsline{toc}{section}{\nameref{sec:theory_and_hypothesis}}

Commercial buildings are complex and encompass a wide variety of purposes. In order to be functional, they all must be powered, and require a considerable amount of energy to operate properly, which can be costly.  In fact, there is a whole industry dedicated to ensuring the proper operation of a structure.  The most direct example is the ASHRAE energy audit process.  As part of the initial audit process, an assessment of the building's overall operational efficiency is gauged.  Typically, an auditor will walk through the buidling, analyze utility bills, and make the closest energy consumption comparisons they can.  This takes years of experience and sometimes requires highly tuned spreadsheets that have been developed over a long period of time.  It can take a surprising amount of effort just to determine if a building is operating efficiently or not, which demonstrates how useful it could be to have a model at hand which predicts building consumption based on easily attainable features.

The CBECS data set provide some insight as to what building attributes most greatly affect building energy consumption.  Over 400 survey questions are recorded and coupled with major fuel consumption.  These fuel sources are Electricity, Natural Gas, District Heat, and Fuel Oil.  However, it would not be useful to construct a model with a large number of predictors, as it would require a large amount of time and effort to compile the necessary information in order to provide a prediction.  Therefore, one of the main focuses for this study will be to extract the fewest amount of predictors necessary in order to make accurate predictions.  