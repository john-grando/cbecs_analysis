{\section*{Problem Statement}}

Commercial buildings are complex and encompass a wide variety of purposes. However, they all must be powered, and require a considerable amount of energy to operate properly.  There are a variety of texts that are dedicated to the analysis of building energy consumption, and determining operating efficiency, such as ASHRAE Guildine 14, etc. etc.....  Particularly, there is a well thought out process for auditing commercial buildings, known as ASHRAE Audits, which start at the lowest level (I) and progress to the highest level (III) as the opportunity for energy and cost savings becomes more apparent (http://aea.us.org/3143-2.html).  As part of the initial audit process, an assessment of the building's overall operational efficiency is gauged.  Typically, an auditor will walk through the buidling, analyze utility bills, and make the closest comparisons they can based on experience.  This takes years of experience and sometimes requires highly tuned spreadsheets that have been developed over years.  It can take a surprising amount of time just to determine if a building is operating efficiently or not.

The CBECS data set provide some insight as to what building attributes most greatly affect building energy consumption.  Over XXX survey questions are recorded and coupled with major fuel consumption.  These fuel sources are Electricity, Natural Gas, District Heat, Fuel Oil, and ...  However, it would not be useful to construct a model with a large number of predictors, as it would require a large amount of time and effort to compile the necessary information in order to provide a prediction.  Therefore, one of the main focuses for this study will be to extract the fewest amount of predictors necessary in order to make accurate predictions.  

Given the complex nature, it is unlikely a linear regression will provide the best prediction accuracy. 