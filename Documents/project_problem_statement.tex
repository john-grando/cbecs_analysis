\section*{Theory and Hypothesis}
\label{sec:theory_and_hypothesis}
\addcontentsline{toc}{section}{\nameref{sec:theory_and_hypothesis}}

Commercial buildings are complex and encompass a wide variety of purposes. However, they all must be powered, and require a considerable amount of energy to operate properly.  The most direct example of this complex issue is the ASHRAE energy audit process.  As part of the initial audit process, an assessment of the building's overall operational efficiency is gauged.  Typically, an auditor will walk through the buidling, analyze utility bills, and make the closest comparisons they can based on experience.  This takes years of experience and sometimes requires highly tuned spreadsheets that have been developed over years.  It can take a surprising amount of time just to determine if a building is operating efficiently or not.

The CBECS data set provide some insight as to what building attributes most greatly affect building energy consumption.  Over 400 survey questions are recorded and coupled with major fuel consumption.  These fuel sources are Electricity, Natural Gas, District Heat, and Fuel Oil.  However, it would not be useful to construct a model with a large number of predictors, as it would require a large amount of time and effort to compile the necessary information in order to provide a prediction.  Therefore, one of the main focuses for this study will be to extract the fewest amount of predictors necessary in order to make accurate predictions.  

Given the complex nature, it is unlikely a linear regression will provide the best prediction accuracy.  This point is especially highlighted by the fact the the goal of this study is produce a parsimonious set of predictors, which means a small subset must be selected.  Therefore, an investigation into more complex, nonlinear, algoriithms will be performed in order to keep the number of necessary predictors as low as possible while still capturing complex interactions.