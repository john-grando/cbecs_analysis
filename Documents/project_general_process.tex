\section{Data and Methods}
\subsection{General Process}

Due to the large number of features in the survey responses, it is not possible to analyze each one individually.  Therefore, the first steps in the process will be centered around selecting a smaller subset from variouse feature extraction algorithms.  The magnitude and contribution percentage of each variable will be considered in selecting features from this model.  Also, the various error rates from each preliminary model will be used as a benchmark for the final model performance.  In order to try and normalize the data, the response variable was divided by the gross floor area of the building and reported in in units of BTU per square foot (e.g. ELBTU/PerSF, NGBTU/PerSF). 

A neural network model will be built to take the verified subset of features and make predictions for the selected major fuel use.  A variety of hyperparameters will be tested, using cross-validation, and compared on a common error metric.  This step will reveal the optimal hyperparameter combination to use for the model.  The prospective model will then be retrained on the entired entire training and validation data.  This model's selected error metrics will then be compared to the preliminary models, which should be considered a floor for performance.  The response variable will als be transformed to units of total consumption and have the resulting metrics re-reported so that an approximation on actual consumption can be made.