{\Large {Background}}
\\
Commercial Building Energy Consumption accounts for approximately 25\% \footnote{\href{https://www.eia.gov/energyexplained/index.php?page=us_energy_commercial}{EIA - \url{https://www.eia.gov/energyexplained/index.php?page=us_energy_commercial}}} of the United States energy production profile.  Many economical and sociological factors are pushing owners of these buildings to reduce energy consumption and optimize performance.  However, it is difficult to say whether a building is operating efficiently or not.  Comparing summary statistics, such as energy use per square foot, is not as simple as it seems because there are a multitude of factors that affect a building’s energy consumption profile.  The complexity of making similar comparisons creates a situation where it is difficult to determine whether a building is performing consistent with, or better than, other standard practice buildings. 
\\
Commercially, ENERGY STAR \footnote{\href{https://www.energystar.gov/}{ENERGY STAR - \url{https://www.energystar.gov/}}} has implemented a benchmarking algorithm that scores buildings on a scale from 1 – 100 using market-available data.  While it is unclear the number of sources used, one is definitely the Commercial Buildings Energy Consumption Survey (CBECS).  The output of this benchmarking algorithm is a unit-less score, as well as a reference ‘baseline’ building; however, the methodology is not released and it is unclear what factors are important to influence the energy consumption of the building.  These barriers make it difficult to provide custom comparisons and nearly impossible to make batch predictions from a set of buildings.
\\[0.125in]
\textbf{Source - Commercial Buildings Energy Consumption Survey (CBECS)}
\\[0.0625in]
CBECS is a national sample survey that collects information on the stock of U.S. commercial buildings, including their energy-related building characteristics and energy usage data (consumption and expenditures). Commercial buildings include all buildings in which at least half of the floorspace is used for a purpose that is not residential, industrial, or agricultural. By this definition, CBECS includes building types that might not traditionally be considered commercial, such as schools, hospitals, correctional institutions, and buildings used for religious worship, in addition to traditional commercial buildings such as stores, restaurants, warehouses, and office buildings\footnote{href{(https://www.eia.gov/consumption/commercial/about.php)}{EIA - \url{https://www.eia.gov/consumption/commercial/about.php}}}.